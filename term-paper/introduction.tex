With the Introduction of Bitcoin in 2008, a signifikant new currency was brought into life.
It is based on a peer to peer network which gives the opportunity to cancel out intermediaries within money transfers, through a decentralized verification process.
With that knowledge the idea was born to use such a technology to run decentralized applications on such a platform.
Ethereum can provide that and is now the second largest cryptocurrency after Bitcoin in terms of market capitalization. []

This paper is supposed to perform an explorative study on the Ethereum blockchain. 
The idea is to replicate what Lischke and Fabian did and transfer it to another crypto technology the Ethereum Blockchain. 
They used the address spaces and transaction history of the Bitcoin blockchain and added off-set network data to get more insights about the bitcoin economy. 
The period analysed included the first four years of the bitcoin existence. 
They were able to combine off-set network data and transaction data to get good statements about the business categories the currency is used for.

This paper is structured as follows. 
Section \ref{ethereum-theory} and \ref{ethereum-blockchain} are dedicated to the basics of Ethereum and how it got evolved from the bitcoin currency, as well as the technology behind it, respectively. 
As we replicate a paper about the bitcoin blockchain [Fabian, 2016] we also compare the Ethereum blockchain with the bitcoin blockchain. We follow this by a brief literature review in section \ref{literature-review}.


After that the reader got a proper foundation that we can dive into the underlying methodology in section \ref{methods}. 
The focus here lies mainly on how we obtained our dataset in subsection \ref{methods-data-preparation} and an overview of the methods used during graph analysis in subsection \ref{methods-graph-analysis}. 
Moving onward we present our findings in section \ref{findings}, where we describe the outcome of our methods for data aquiration in subsection \ref{findings-data-preparation}, structure them using traditional statistical methods in subsection \ref{findings-descriptive} and graph analysis in subsection \ref{findings-graph-analysis}.
Finally, section \ref{conclusion} concludes the article and discusses the outlook we received during our analysis. 
We also want to use this part to discuss what are the main differences to the former paper about the bitcoin blockchain and give an idea about future work. 
