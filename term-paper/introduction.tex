With the Introduction of Bitcoin in 2008, a significant new currency was brought into life.
It is based on a peer to peer network which gives the opportunity to cancel out intermediaries within money transfers, through a decentralized verification process. 
With that knowledge, the idea was born to use such a technology to also run decentralized applications, which can execute whole programs on such a platform. 
Ethereum can provide that and is now the second largest cryptocurrency after Bitcoin in terms of market capitalization \cite{BitInfoEther, BitInfoBitcoin}.
This paper proposes an explorative study on the Ethereum blockchain. 
The idea is to replicate what \cite{lischke2016analyzing} did and transfer it to another crypto technology - the Ethereum Blockchain. 
They used the address spaces and transaction history of the Bitcoin and added off-set network data to get more insights about the Bitcoin economy. 
The analyzed period included the first four years of the Bitcoin existence. They were able to combine off-set network data and transaction data to get good statements about the business categories the currency is used for. 
With the Ethereum we try to figure out if we are able to collect the transaction data and IP addresses to get more insights about the crypto technology. Specifically, we want to figure out how much information one is able to get out of the transaction and off-set data to get deeper insights into what is going on in the network. Furthermore, we also want to assess if the information discloses security problems for the network. 
The period is shorter than in the replicated paper, but we do use the same resources.

% This paper is structured as follows. The first part is dedicated to the basics of Ethereum and how it got evolved from the Bitcoin currency, or moreover the technology behind it. As we replicate a paper about the Bitcoin blockchain \cite{lischke2016analyzing} we also compare the Ethereum blockchain with the Bitcoin blockchain. Give some relevant economic data about both.
This paper is structured as follows. 
Section \ref{ethereum-theory} and \ref{ethereum-blockchain} are dedicated to the basics of Ethereum and how it got evolved from the Bitcoin currency, as well as the technology behind it, respectively. 
As we replicate a paper about the Bitcoin blockchain \cite{lischke2016analyzing} we also compare the Ethereum blockchain with the Bitcoin blockchain. We follow this by a literature review in section \ref{literature-review}.
Afterwards, we follow by demonstrating the underlying methodology in section \ref{methods}. 
The focus here lies mainly on how we obtained our dataset in subsection \ref{methods-data-preparation} and an overview of the methods used during descriptive and graph analysis in subsection \ref{methods-descriptive} and \ref{methods-graph-analysis}. 
Moving onward, we present our findings in section \ref{findings}, where we describe the outcome of our methods for data acquisition in subsection \ref{findings-data-preparation}, structure them using traditional statistical methods in subsection \ref{findings-descriptive} and discuss graph analysis in subsection \ref{findings-graph-analysis}.
Finally, section \ref{conclusion} concludes the article and discusses the findings of our research. 
We also want to use this part to discuss the main differences to the former paper about the Bitcoin blockchain and give an idea about future work.
