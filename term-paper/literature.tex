We aspire to adopt the approaches towards the study of blockchain described in \cite{lischke2016analyzing}. The main difference of our research is that it targets ethereum instead of the bitcoin. For this reason we pay special attention to the papers focused on ethereum blockchain. Thus, \cmmnt{Payette et al. (2017)} \cite{payette2017characterizing} research ethereum address space; \cmmnt{Chan {\&} Olmsted (2017)} \cite{chan2017ethereum} and \cmmnt{Chen et al. (2018)} \cite{chen2018understanding} perform ethereum graph analysis; \cmmnt{Li et al (2017)} \cite{li2017etherql} develop a query layer for ethereum blockchain; \cmmnt{Anoaica \& Levard (2018)} \cite{anoaica2018quantitative} perform quantitative description of internal activity on the ethereum blockchain; \cmmnt{Somin, Gordon \& Altshuler (2018)} \cite{somin2018social} study the dynamics of the ”social signals” of ethereum network, provide insights about the ecosystem and the forces acting within it and demonstrate that the network displays strong power-law properties. 

We also go through the generic information about ethereum: including its white \cmmnt{(Buterin (2014)} \cite{buterin2014next}) and yellow (\cmmnt{Wood, G. (2014)}\cite{wood2014ethereum} papers), comparative studies of crypto-currencies (see for instance \cmmnt{Maesa (2018)} \cite{maesa2018blockchain}, \cmmnt{Rudlang, M. (2017)} \cite{rudlang2017comparative}, \cmmnt{Sapuric et al. (2017)} \cite{sapuric2017distributed}, \cmmnt{Anderson et al. (2016)} \cite{anderson2016new}). Given the role that smart-contracts play in ethereum blockchain we also consider research on this topic. For instance: \cmmnt{Grishchenko et al. (2018)} \cite{grishchenko2018semantic} do semantic analysis of ethereum smart-contracts; \cmmnt{Tikhomirov et al. (2018)} \cite{tikhomirov2018smartcheck} come up with tools for their static analysis; \cmmnt{Bartoletti \& Pompianu (2017)} \cite{bartoletti2017dissecting} research practical applications of smart contracts and their design patterns; \cmmnt{Bartolletti et al. (2017)} \cite{bartoletti2017dissecting} suggest ways to identify Ponzi schemes.

To complete the picture we also consider research on network analysis of bitcoin blockchain that were suggested in \cmmnt{Lischke \& Fabian (2016)} \cite{lischke2016analyzing} and can provide some additional inspiration: \cmmnt{Reid \& Harrigan (2013)} \cite{reid2013analysis}, \cmmnt{Baumann, Fabian, Lischke (2014)} \cite{baumann2014exploring}, \cmmnt{Drainville (2012)} \cite{drainville2012analysis}, \cmmnt{Ober et al. (2013)} \cite{ober2013structure}, \cmmnt{Meiklejohn et al. (2013)} \cite{meiklejohn2013fistful}, \cmmnt{Spagnuolo et al. (2014)} \cite{spagnuolo2014bitiodine}, \cmmnt{Androulaki (2013)} \cite{androulaki2013evaluating}, \cmmnt{Kaminsky \& Black (2011)} \cite{kaminsky2011black}, \cmmnt{Ortega (2013)} \cite{ortega2013bitcoin}.
